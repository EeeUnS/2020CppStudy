
\documentclass[10pt]{beamer}
\usepackage{kotex}

\usepackage{framed}
\usepackage{graphicx}
%https://www.overleaf.com/learn/latex/Inserting_Images

\usepackage{amsmath}
%use dfrac
\usepackage{xcolor}

\usepackage{amsthm}
%\usepackage{tabl}
\usepackage{listings}
\definecolor{mGreen}{rgb}{0,0.6,0}
\definecolor{mGray}{rgb}{0.5,0.5,0.5}
\definecolor{mPurple}{rgb}{0.58,0,0.82}
\definecolor{backgroundColour}{rgb}{0.95,0.95,0.92}
%https://tex.stackexchange.com/questions/348651/c-code-to-add-in-the-document
\lstdefinestyle{CStyle}{
    backgroundcolor=\color{backgroundColour},   
    commentstyle=\color{mGreen},
    keywordstyle=\color{magenta},
    numberstyle=\tiny\color{mGray},
    stringstyle=\color{mPurple},
    basicstyle=\footnotesize,
    breakatwhitespace=false,         
    breaklines=true,                 
    captionpos=b,                    
    keepspaces=true,                 
    numbers=left,                    
    numbersep=5pt,                  
    showspaces=false,                
    showstringspaces=false,
    showtabs=false,                  
    tabsize=2,
    language=C
}

\usepackage{url}

\usepackage{etoolbox}
\AtBeginEnvironment{quote}{\singlespacing\small}


\usepackage{thmtools}
\usepackage{xcolor}
\declaretheoremstyle[% spaceabove=6pt,spacebelow=6pt, headfont=\color{MainColorOne}\sffamily\bfseries, notefont=\mdseries, notebraces={[}{]}, bodyfont=\normalfont,
headpunct={},
postheadspace=1em,
%qed=▣,
]{maintheorem}

\declaretheorem[%
name=정의,
style=maintheorem,
numberwithin=section, shaded={%bgcolor=MainColorThree!20,
margin=.5em}]{dfn}
% \begin{dfn}[]
% \end{dfn}

\setbeamertemplate{footline}[frame number]

\usetheme{Hannover}


\title{Class}

\author{EUnS}

\begin{document}


\begin{frame}
    \maketitle
\end{frame}    

\begin{frame}{}
    \tableofcontents
\end{frame}   

\section{OOP}
\begin{frame}{OOP란?}
    \begin{itemize}
        \item 객체 지향 프로그래밍(Object oriented proggraming)
        \item 사실 본인도 잘모름
        \item 어떤 물체를 클래스 설정하고 얘를 메소드(멤버 함수)로 가지고노는것.
        \item 큰 단위의 프로젝트의 협업을 위한것(큰 코드를 다룰때) : 추상화
        \item 그럼에도 클래스 내부를 이해하는것은 중요하다.
    \end{itemize}
\end{frame}    


\section{class}

\begin{frame}[fragile]{class}
    \begin{lstlisting}[style = CStyle]
    class name
    {
    public:
        name();
        ~name();
        void function1();
        void function2();
        void function3();
    private:
        int memberVariable1;
        char memberVariable2;
        void function4();
    };
    void name::fuction1() {    }
    \end{lstlisting}
\end{frame}    

\begin{frame}{용어 정리}
    \begin{itemize}
        \item 멤버 함수(member variable)
        \item 메소드(method) = 멤버 함수(member function)
        \item public : 외부에서 자유롭게 사용할수있는 것들
        \item private : class내 public에서 접근가능
        \item 객체 = 오브젝트(object) = 인스턴스(instance) : 데이터 할당된 class를 지칭
        \item 캡슐화(encapsulation) : 대충 내부구현을 숨긴다는 뜻
        \item 정보 은닉 : 불필요한 class 멤버 접근을 제한하는것 private으로 넣는다.
        \item 추상화 : 실제 구현을 감싸는것.
        \item protect : 상속 받은 자식에서도 접근가능
        \item 다형성 : 상속을 통한 다양함을 나타낸다는 OOP 특징
        \item 오버라이딩 : 같은 이름의 메소드를 상속 받는것
    \end{itemize}
\end{frame}

\begin{frame}[fragile]{생성자 소멸자}
    \begin{itemize}
        \item 초기화 : 멤버 이니셜라이저
        \item 반환값 X
    \end{itemize}
    
    \begin{lstlisting}[style = CStyle]
    class name
    {
    public:
        name() : memberVariable1(0), memberVariable2('a'){;}
        ~name();
    private:
        int memberVariable1;
        char memberVariable2;
        static int n = 0 ;
    };
    \end{lstlisting}
\end{frame}    




\begin{frame}{C vs Cpp struct, class와 차이}
    \begin{enumerate}
        \item Cpp에서는 struct에 메소드 선언이 가능하다.
        \item C방식으로 struct 안붙이고 뒤에 이름만 사용가능
        \item 기본접근자가 public임 class는 private
        \item Coding standard : struct는 C 스타일로만 사용
    \end{enumerate}
\end{frame}    


\section{operator}


\begin{frame}[fragile]{operator}
    \begin{enumerate}
        \item class의 연산을 일일이 지정해줘야됨
        \item \href{https://en.wikipedia.org/wiki/Operators_in_C_and_C%2B%2B}{\textcolor{blue}{참고}}
        여기에서보고 필요한거 만들어씀
        

    \end{enumerate}


    \begin{lstlisting}[style = CStyle]
    class vector
    {
    public:
        vector operator+(const vector& b);
    private:
        int x, y;
    };

    vector vector::operator+(const vector& b)
    {
        vector c;
        c.x = b.x + x;
        c.y = b.y + y;
        return c;
    }
    \end{lstlisting}
    
\end{frame}    


\section{memory}

\begin{frame}{class 메모리 구조}
    \begin{enumerate}
        \item 변수 선언순으로 올라감.
        \item 함수 .text 영역에
        \item static 변수는 .BSS 영역
        \item 바이트 패딩 일어남
    \end{enumerate}
\end{frame}    


\section{Copy}

\begin{frame}[fragile]{Copy}
    \begin{lstlisting}[style = CStyle]
        class name
        {
        public:
            name();
            ~name();
        private:
            int memberVariable1;
        };
        name a();
        name b = a; 
        \end{lstlisting}
\end{frame}    


\begin{frame}{Copy \& move}
    \begin{itemize}
        \item 클래스 생성시 몇가지 정의가 자동으로 생성된다.
        \item 기본생성되는것들
        \begin{itemize}
            \item default constuctor
            \item default destuctor
            \item copy constuctor 
            \item move constructor
            \item copy assignment operator
            \item move assignment operator
        \end{itemize}
        \item $=default, =delete$ 로 명시할수있다.
    \end{itemize}
\end{frame}    


\begin{frame}[fragile]{Copy}
    \begin{lstlisting}[style = CStyle]
        class name
        {
        public:
        private:
            int memberVariable1;
        };

        class name
        {
        public:
            name() = default;
            ~name() = default;
            name(const name&) = default;
            name& operator=(const name&) = default;
            name(name&&) = default;
            name& operator=(name&&) = default;
        private:
            int memberVariable1;
        };
        \end{lstlisting}
        default copy와 move는 단순히 값을 복사하기만함.
\end{frame}



\begin{frame}[fragile]{Copy}
    \begin{lstlisting}[style = CStyle]
    class name
    {
    public:
    name():mString = nullptr;
    {
        mString = new char[20];
        ...
    }
    private:
        char* mString;
    };
    name a;
    name b = a;
    \end{lstlisting}
    주소를 한번 찾아보자.
\end{frame}



\begin{frame}[fragile] {move (C++11)}
    \begin{itemize}
        \item rvalue vs lvalue
        \item 너무 어려워요
        \item 필요하면 그때..
    \end{itemize}
    \begin{lstlisting}[style = CStyle]
        name a ,b;
        ...
        name c = a*b;
    \end{lstlisting}
\end{frame}

\begin{frame}{과제}
    과제 1
    
    auto, using namespace std; 쓰지말것
\end{frame}    


\end{document}

%%%%%%%%%%%%%%%%%%%%%%%%%%%%%%%%%%%%%%%%%%%%%%%%%%

% \begin{frame}{}
% \end{frame}    

% \begin{frame}[fragile]{}
        
%     \begin{lstlisting}[style = CStyle]
%     \end{lstlisting}
% \end{frame}    


