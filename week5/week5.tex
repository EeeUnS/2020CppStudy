
\documentclass[10pt]{beamer}
\usepackage{kotex}

\usepackage{framed}
\usepackage{graphicx}
%https://www.overleaf.com/learn/latex/Inserting_Images

\usepackage{amsmath}
%use dfrac
\usepackage{xcolor}

\usepackage{amsthm}
%\usepackage{tabl}
\usepackage{listings}
\definecolor{mGreen}{rgb}{0,0.6,0}
\definecolor{mGray}{rgb}{0.5,0.5,0.5}
\definecolor{mPurple}{rgb}{0.58,0,0.82}
\definecolor{backgroundColour}{rgb}{0.95,0.95,0.92}
%https://tex.stackexchange.com/questions/348651/c-code-to-add-in-the-document
\lstdefinestyle{CppStyle}{
    backgroundcolor=\color{backgroundColour},   
    commentstyle=\color{mGreen},
    keywordstyle=\color{magenta},
    numberstyle=\tiny\color{mGray},
    stringstyle=\color{mPurple},
    basicstyle=\footnotesize,
    breakatwhitespace=false,         
    breaklines=true,                 
    captionpos=b,                    
    keepspaces=true,                 
    numbers=left,                    
    numbersep=5pt,                  
    showspaces=false,                
    showstringspaces=false,
    showtabs=false,                  
    tabsize=2,
    language=C++
}

\usepackage{url}

\usepackage{etoolbox}
\AtBeginEnvironment{quote}{\singlespacing\small}


\usepackage{thmtools}
\usepackage{xcolor}
\declaretheoremstyle[% spaceabove=6pt,spacebelow=6pt, headfont=\color{MainColorOne}\sffamily\bfseries, notefont=\mdseries, notebraces={[}{]}, bodyfont=\normalfont,
headpunct={},
postheadspace=1em,
%qed=▣,
]{maintheorem}

\declaretheorem[%
name=정의,
style=maintheorem,
numberwithin=section, shaded={%bgcolor=MainColorThree!20,
margin=.5em}]{dfn}
% \begin{dfn}[]
% \end{dfn}

\setbeamertemplate{footline}[frame number]

\usetheme{Hannover}


\title{week5 후에 공부할것들.}

\author{EUnS}

\begin{document}


\begin{frame}
    \tableofcontents
\end{frame}    

\section{과목}

\begin{frame}{과목}
    \begin{itemize}
        \item computer architecture
        \item os
        \item etc
        \begin{itemize}
            \item system programming
            \item socket programming
            \item parrell programming
            \item graphicx
        \end{itemize}
    \end{itemize}
\end{frame}

\begin{frame}{과목}
    \begin{itemize}
        \item computer architecture : csapp, computer architecture a quantitative approach
        \item os : 공룡책
        \item etc
        \begin{itemize}
            \item system programming : csapp
            \item socket programming 
            \item parrell programming : Cpp Concurrency In Action\href{https://github.com/CppKorea/CppConcurrencyInAction}{\textcolor{blue}{참고}}
            \item graphicx
        \end{itemize}
    \end{itemize}
\end{frame}



\begin{frame}{c++}
    \begin{itemize}
        \item c++ programming language(Bjarne Stroustrup)
        \item a tour of c++(Bjarne Stroustrup)
        \item (modern)effective c++ (scott-meyer)
    \end{itemize}
    etc..

    \begin{itemize}
        \item \href{https://github.com/isocpp/CppCoreGuidelines/blob/master/CppCoreGuidelines.md}{\textcolor{blue}{cpp가이드라인}},
        \href{https://github.com/CppKorea/CppCoreGuidelines}{\textcolor{blue}{번역}}
        \item \href{https://github.com/utilForever/game-developer-roadmap}{\textcolor{blue}{게임개발로드맵}}
        \item \href{https://en.cppreference.com/w/}{\textcolor{blue}{cppref}}
    \end{itemize}
\end{frame}


%%%%%%%%%%%%%%%%%%%%%%%%%%class 세부사항
\section{class}


\begin{frame}{class}
    \begin{itemize}
        \item reference 반환을 할때?
        \item friend : 특정 함수에대해서 해당 클래스의 private영역을 사용할 수 있게 허락해줌
    \end{itemize}
\end{frame}



\begin{frame}[fragile]{ex}
    \begin{lstlisting}[style = CppStyle]
    class sample
    {
    public: 
        sample(const sample& a) = default;
        sample& operator=(const sample& a) = default;
        
        sample& operator+(const sample& c)
        { 
            a += c.a; b += c.b;
            return c;
        }
    private:
        int a;
        int b; 
    }
    sample a(1,20),b(29,30);
    a = b; //compile ok default fucntion
    a = b = c; // ?
    a = b + c;
    \end{lstlisting}
\end{frame}    


\begin{frame}
    a.operator=(b.operator=(c))

\end{frame}




\begin{frame}[fragile]{I/O setting}
    \begin{lstlisting}[style = CppStyle]
    #include<istream>
    #include<ostream>

    class sample
    {
    public:
        friend std::ostream& operator<<(std::ostream& os, const sample& a);
        friend std::istream& operator>>(std::istream& os, sample& a);
    private:
        int a;
        int b; 
    }
        
    std::ostream& operator<<(std::ostream& os, const sample& c)
    {
        os << a <<' ' << b;
        return os;
    }

    std::istream& operator>>(std::istream& os, sample& c)
    {
        os >> a >> b;
        return os;
    }
    \end{lstlisting}
\end{frame}    

%%%%%%%%%%%%%%%%%%%%%%%%%%value
\section{value}

\begin{frame}{value}
    \begin{itemize}
        \item lvalue
        \item rvalue
    \end{itemize}
\end{frame}


\begin{frame}[fragile]{기초 개념}
    \begin{lstlisting}[style = CppStyle]
        sample(sample&& a) = default;
        sample& operator=(sample&& a) = default;
    \end{lstlisting}
\end{frame}

\begin{frame}{value(C++11)}
    \href{https://en.cppreference.com/w/cpp/language/value_category}{\textcolor{blue}{참고}}
    
    \begin{itemize}
        \item lvalue
        \item pvalue
        \item prvalue
        \item xvalue
    \end{itemize}

    나도 잘 모르니 생략.
\end{frame}


\begin{frame}{forward}
    \begin{itemize}
        \item $std::move$
        \item move semantics
        \item perfect forwarding
        \item universe reference : scott meyers가 처음으로 명명함. 정식 명칭은 아니나 워낙 유명해서 표준처럼 불림.\href{https://isocpp.org/blog/2012/11/universal-references-in-c11-scott-meyers}{\textcolor{blue}{원글}}, \href{http://egloos.zum.com/sweeper/v/3149089}{\textcolor{blue}{번역}}
        
    \end{itemize}
\end{frame}





%%%%%%%%%%%%%%%%%%%%%%%%%%networking
\section{networking}


\begin{frame}{}
    c++ 표준에는 아직 없음.
    \begin{itemize}
        \item boost asio : 크로스 플랫폼
        \item winsock
        \item BSD
    \end{itemize}
    
    \href{https://jacking.tistory.com/1267}{\textcolor{blue}{참고}}
    
\end{frame}


% \begin{frame}[fragile]{}
%     \begin{lstlisting}[style = CppStyle]
    
%     \end{lstlisting}
% \end{frame}    


% \begin{frame}{}
%     \begin{figure}[h!]
%         %\centering
%         \includegraphics[scale=0.25]{}
%         \caption{}
%     \end{figure}
% \end{frame}    




%%%%%%%%%%%%%%%%%%%%%%%%%%multithreading c++17,20

%atomic, mutex, task기반


%%%%%%%%%%%%%%%%%%%%%%%%%%api 



\end{document}



% \begin{frame}{}
%     \href{}{\textcolor{blue}{참고}}
% \end{frame}    



% \begin{frame}{}
%     \begin{itemize}
%         \item 
%         \item 
%         \item 
%     \end{itemize}
% \end{frame}


% \begin{frame}[fragile]{}
%     \begin{lstlisting}[style = CppStyle]
    
%     \end{lstlisting}
% \end{frame}    


% \begin{frame}{}
%     \begin{figure}[h!]
%         %\centering
%         \includegraphics[scale=0.25]{}
%         \caption{}
%     \end{figure}
% \end{frame}    


