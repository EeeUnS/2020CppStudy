
\documentclass[10pt]{beamer}
\usepackage{kotex}

\usepackage{framed}
\usepackage{graphicx}
%https://www.overleaf.com/learn/latex/Inserting_Images

\usepackage{amsmath}
%use dfrac
\usepackage{xcolor}

\usepackage{amsthm}
%\usepackage{tabl}
\usepackage{listings}
\definecolor{mGreen}{rgb}{0,0.6,0}
\definecolor{mGray}{rgb}{0.5,0.5,0.5}
\definecolor{mPurple}{rgb}{0.58,0,0.82}
\definecolor{backgroundColour}{rgb}{0.95,0.95,0.92}
%https://tex.stackexchange.com/questions/348651/c-code-to-add-in-the-document
\lstdefinestyle{CStyle}{
    backgroundcolor=\color{backgroundColour},   
    commentstyle=\color{mGreen},
    keywordstyle=\color{magenta},
    numberstyle=\tiny\color{mGray},
    stringstyle=\color{mPurple},
    basicstyle=\footnotesize,
    breakatwhitespace=false,         
    breaklines=true,                 
    captionpos=b,                    
    keepspaces=true,                 
    numbers=left,                    
    numbersep=5pt,                  
    showspaces=false,                
    showstringspaces=false,
    showtabs=false,                  
    tabsize=2,
    language=C
}

\usepackage{url}

\usepackage{etoolbox}
\AtBeginEnvironment{quote}{\singlespacing\small}


\usepackage{thmtools}
\usepackage{xcolor}
\declaretheoremstyle[% spaceabove=6pt,spacebelow=6pt, headfont=\color{MainColorOne}\sffamily\bfseries, notefont=\mdseries, notebraces={[}{]}, bodyfont=\normalfont,
headpunct={},
postheadspace=1em,
%qed=▣,
]{maintheorem}

\declaretheorem[%
name=정의,
style=maintheorem,
numberwithin=section, shaded={%bgcolor=MainColorThree!20,
margin=.5em}]{dfn}
% \begin{dfn}[]
% \end{dfn}

\setbeamertemplate{footline}[frame number]

\usetheme{Hannover}


\title{template / linker}

\author{EUnS}

\begin{document}

%linker
% #include이해
%template
% TMP맛보기
% 과제 3-2

\begin{frame}{}
    \maketitle
\end{frame}    

\begin{frame}{}
    \tableofcontents
\end{frame}   

\section{linker}
\begin{frame}{}
\end{frame}    

\begin{frame}{}
\end{frame}    

\begin{frame}{}
\end{frame}    

\begin{frame}{}
\end{frame}    

\begin{frame}{}
\end{frame}    

\begin{frame}{}
\end{frame}    

\begin{frame}{}
\end{frame}    

\begin{frame}{}
\end{frame}    

\begin{frame}{}
\end{frame}    

\begin{frame}{}
\end{frame}    

\begin{frame}{}
\end{frame}    

\begin{frame}{}
\end{frame}    

\begin{frame}{}
\end{frame}    

\begin{frame}{}
\end{frame}    

\begin{frame}{}
\end{frame}    


\section{#include이해}

\begin{frame}{header file}
    \begin{itemize}
        \item 그냥 텍스트파일과 같다.
        \item 선언을 위한것...
        \item 실제 구현은 어느 cpp파일에서...
        \item #include로 cpp파일에 그대로 붙여넣는다.
    \end{itemize}
\end{frame}

\section{template}

\begin{frame}{template}
    \begin{itemize}
        \item template을 실제로 인자를 넣어서 사용할때 전처리에서 인자를 넣은 코드를 생성함
        \item 타입 아무거나 넣어도 가능
        \item 실행 파일크기가 예상과 다르게 커질수있음
    \end{itemize}
\end{frame}    


\begin{frame}[fragile]{}
    \begin{lstlisting}[style = CStyle]
        template<typename T>
        T function(T a , T* b, T& c)
        {;}
        template<>
        char function<char>(char a , char* b, char &c)
        {;}

        int main()
        {
            function(1,0x0000,3);
            function("s", 0x0000,"asdf");
        }
    \end{lstlisting}
\end{frame}    



\begin{frame}{}
\end{frame}    

\begin{frame}{}
\end{frame}    

\begin{frame}{}
\end{frame}    

\begin{frame}{}
\end{frame}    


\section{TMP}

\begin{frame}[fragile]{TMP}
    
    \begin{lstlisting}[style = CStyle]
        
    \end{lstlisting}
    
    \begin{itemize}
        \item template의 특성을 이용해서 반복되는 계산을 컴파일타임에 계산을 해놓은다음 그 값을 $O(1)$에 부르는 흑마법
        \item \href{https://libsora.so/posts/friday-the-13th-tmp/}{\textcolor{이런짓도 가능}
    \end{itemize}
    
\end{frame}    

\begin{frame}{TMP는 나쁘다}
    \href{}{\textcolor{참고}
    \begin{itemize}
        \item 극암의 코드 가독성
    \end{itemize}
\end{frame}    


\section{constexpr (c++11)}
\begin{frame}{constexpr}
    \begin{itemize}
        \item 변수에 사용할때
        \begin{itemize}
            \item #define 대체가능
            \item const 상수 대체가능
            \item 컴파일타임에 상수로 대체
        \end{itemize}
        \item 함수에 사용할때
        \begin{itemize}
            \item 어느부분 TMP 대체가능
        \end{itemize}
    \end{itemize}
\end{frame}    
% 과제 3-2



\begin{frame}{과제}
    과제 3-2 : 링크드리스트
\end{frame}    


\end{document}


% \begin{frame}{}
% \end{frame}    
%   \href{https://www.youtube.com/watch?v=OMiEwfmfdng&feature=youtu.be}{\textcolor{blue}
% \begin{frame}[fragile]{}
        
%     \begin{lstlisting}[style = CStyle]
    
%     \end{lstlisting}

% \end{frame}    
