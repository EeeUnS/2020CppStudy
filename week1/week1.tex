
\documentclass[10pt]{beamer}
\usepackage{kotex}

\usepackage{framed}
\usepackage{graphicx}
%https://www.overleaf.com/learn/latex/Inserting_Images
\usepackage{color}
\usepackage{amsmath}
%use dfrac
\usepackage{xcolor}

\usepackage{amsthm}
%\usepackage{tabl}
\usepackage{listings}
\definecolor{mGreen}{rgb}{0,0.6,0}
\definecolor{mGray}{rgb}{0.5,0.5,0.5}
\definecolor{mPurple}{rgb}{0.58,0,0.82}
\definecolor{backgroundColour}{rgb}{0.95,0.95,0.92}
%https://tex.stackexchange.com/questions/348651/c-code-to-add-in-the-document
\lstdefinestyle{CStyle}{
    backgroundcolor=\color{backgroundColour},   
    commentstyle=\color{mGreen},
    keywordstyle=\color{magenta},
    numberstyle=\tiny\color{mGray},
    stringstyle=\color{mPurple},
    basicstyle=\footnotesize,
    breakatwhitespace=false,         
    breaklines=true,                 
    captionpos=b,                    
    keepspaces=true,                 
    numbers=left,                    
    numbersep=5pt,                  
    showspaces=false,                
    showstringspaces=false,
    showtabs=false,                  
    tabsize=2,
    language=C
}

\usepackage{url}

\usepackage{etoolbox}
\AtBeginEnvironment{quote}{\singlespacing\small}


\usepackage{thmtools}
\usepackage{xcolor}
\declaretheoremstyle[% spaceabove=6pt,spacebelow=6pt, headfont=\color{MainColorOne}\sffamily\bfseries, notefont=\mdseries, notebraces={[}{]}, bodyfont=\normalfont,
headpunct={},
postheadspace=1em,
%qed=▣,
]{maintheorem}

\declaretheorem[%
name=정의,
style=maintheorem,
numberwithin=section, shaded={%bgcolor=MainColorThree!20,
margin=.5em}]{dfn}
% \begin{dfn}[]
% \end{dfn}

\setbeamertemplate{footline}[frame number]

\usetheme{Hannover}


\title{Cpp 스터디 전체적인 개요}

\author{EUnS}

\begin{document}

% 병렬프로그래밍 ,네트워크는 다루지않음

\begin{frame}
    \maketitle
\end{frame}    

\begin{frame}
    \tableofcontents

\end{frame}    

\section{C++ 소개}

\begin{frame}{C++ 소개}
    \begin{itemize}
        \item Bjarne Stroustrup이 1979년에 창시
        \item C + class
        \item The C++ Programming Language이 책이 가장 유명함
        %Bjarne Stroustrup
    \end{itemize}
\end{frame}

\begin{frame}{C++ 표준}
    \begin{itemize}
        \item C++ 98,03
        \item C++ 11/14(modern C++)
        \item C++ 17/20~
    \end{itemize}
    초기에는 C가 완벽하게 C++에 포함이 되었는데 C99표준으로부터 더이상 C와는 별개의 언어가 되었음 참고로 마소 컴파일러는 C99를 제대로 지원하지 않음
\end{frame}

\section{C++로서 C와의 차이}

\begin{frame}[fragile]{call by ref}
    C와의 중요한 차이점
    \begin{lstlisting}[style = CStyle]
        int a = 10;
        int &a =20;
        void function(const int &a)
    \end{lstlisting}
\end{frame}


% \begin{frame}{exception}
%     unique_ptr
%     https://www.youtube.com/watch?v=MGVSPZoOchE&list=PLW_uvsSPlijtSmrhajc3Y02G86lOieQOb&index=3
% \end{frame}    

\begin{frame}[fragile]{new delete}
    \begin{itemize}
        \item malloc $\rightarrow$ new type()
        \item free $\rightarrow$ delete 
    \end{itemize}
    malloc은 그저 메모리 할당만을 함. new는 class의 메모리할당 처리또한 같이해주기 때문에 new사용이 필수
    \begin{lstlisting}[style = CStyle]
        char* n = new char[20];
        delete[] n;
    \end{lstlisting}
\end{frame}


\begin{frame}[fragile]{Range Based For(c++11)}
    \href{https://www.youtube.com/watch?v=sVoz36DYK5s&list=PLW_uvsSPlijtSmrhajc3Y02G86lOieQOb}{\textcolor{blue}{참고}}
    
    \begin{lstlisting}[style = CStyle]
    int fibonacci[] = { 0, 1, 1, 2, 3, 5, 8, 13, 21, 34, 55, 89 }; 
    for (int number : fibonacci)
    {
        std::cout << number << ' ';
    }
    std::vector<int> fibonacci = { 0, 1, 1, 2, 3, 5, 8, 13, 21, 34, 55, 89 }; 
    for (const auto& number : fibonacci) 
    {
        std::cout << number << ' ';
    }
    \end{lstlisting}

\end{frame}    

\section{study}


\begin{frame}[fragile] 
    \frametitle{스터디 목표.}
    \begin{itemize}
        \item 구글링 몸에 배기
        \item Cpp 11,14...간단하게 익히기
        \item 객체지향프로그래밍
    \end{itemize}
    솔직히 어느 대학교에선 수업에서 커버 할 수 있는 수준
\end{frame}

\begin{frame}{스터디 전체 개요}
    \begin{enumerate}
        \item 오늘
        \item 클래스
        \item 템플릿 \& 템플릿 클래스
        \item 상속, 가상함수
        \item 미정
    \end{enumerate}
\end{frame}

\begin{frame}{진행방식}
    인원을 확정해서 slack방 새로 만들생각

    \begin{itemize}
        \item 매주 월 10시
        \item code review
        \item 간단한 강의
        \item \href{https://github.com/EeeUnS/2020CppStudy}{\textcolor{blue}{repo}}를 fork weekn/구분할수있는것/파일들로 만든후 pull request
    \end{itemize}



    과제를 하면서 모르겠는 부분은 구글링을 통하거나 책을 통해서 각자 습득


\end{frame}    

\begin{frame}{coding standards}
    \href{https://github.com/popekim/dev-docs-ko/blob/master/coding-standards/cpp.md}{\textcolor{blue}{참고}}

    
    쓰지말것
    \begin{itemize}
        \item using namespace  std;
        \item auto
    \end{itemize}
\end{frame}    


\begin{frame}{Q\&A}
    질문?
\end{frame}


\begin{frame}{과제}
    \begin{itemize}
        \item 2-2 vector 연습
        \item 3-1 sort 연습 
        \item 3-2 범위기반 for문
        \item 3-3 c++11 random class 사용해보기
        \item 4-1 그래프 표현
    \end{itemize}
\end{frame}    

\end{document}



% \begin{frame}{}


% \end{frame}    
% \begin{frame}{}


% \end{frame}    
% \begin{frame}{}


% \end{frame}    
% \begin{frame}{}[fragile] 


% \end{frame}    
